\documentclass[12pt]{book}
        \usepackage{xunicode}
        \usepackage{polyglossia}
	    \setmainlanguage {french}
    		\setotherlanguage{latin}
    		\renewenvironment{latin}
    	{\begin{hyphenrules}{latin}}
    	{\end{hyphenrules}}
        \usepackage{hyperref}
        \usepackage[noend,series={A,B},noeledsec, noledgroup]{reledmac}
            \Xarrangement[A]{paragraph}
        \usepackage{reledpar}
        \usepackage {lineno}
        \usepackage[a4paper]{geometry}
	       \geometry{margin=3.5cm}
	    \usepackage[onehalfspacing]{setspace}

        \begin{document}
        \title{Partie V: arbitrage tiré de Joachim Vadianus, \textit{Gallus Pugnans}, 1514}
        \author{Atelier de traduction du Groupe de théâtre antique\\
            Université de Neuchâtel, CLAM\\
            C. Aeby, N. Aeby, M. Cario, M. Durham,\\ 
            P. Jacsont, S. Moy, I. Muminovic, É. Paupe,\\
            J. Rafael Ribeiro da Silva, R. Richard. P. Schwab}
        \date{Semestre de printemps 2020}
        
        \maketitle
        \begin{pages}
        \begin{latin}
        \begin{Leftside}
        \beginnumbering 
            \pstart\section*{Caporum arbitrium Nomothetes interpres loquitur.}\pend\pstart\subsection*{Nomothetes}\pend\pstart\textbf{Nomothetes}\hspace{1cm} 
                    Recensui, ut jussistis Philonice, tuque Euthyme breviter \ampersand\ enucleate rem omnem Capis, 
                    qui nescio an alia in re \edtext{umquam}{\Afootnote{unquam}} adeo stupefacti fuerint, 
                    ac hodie fuere, postquam illam inter Gallos, Gallinae que audivere \edtext{dissensionem}{\Afootnote{dissentionem}}. 
                    Sciunt equidem, intestinis bellis nil pestilentius esse, \ampersand\ eius \edtext{reipublicae}{\Afootnote{rei publicae}} fulcimenta durare diu non posse, 
                    qua a capite membra dissentiant, \ampersand\ alioqui, veritatis amator Historicus, Concordia dixit res parvas crescere, Discordia maximas dilabi. 
                    Id quod adeo vobis est notum ut quod maxime. 
                    Quapropter eis ne breviter accepta, 
                    visum est eo festinantius ad moderati arbitrii praescriptum contendendum esse quanto res in se plus habitura periculi est, 
                    si longius proteletur, Gratiasque agunt bonae forti, \ampersand\ secundo fato quod arbitrii ad eos libertas, vestro, \ampersand\ partium consensu pervenit. 
                    Si enim res cessisset aliter, timendum ipsis videbatur, 
                    \edtext{ne quis}{\Afootnote{nequis}} immitius quam huiusmodi generis imbecilla securitas commeruisset, agere fuerit ausus, quod hercle nemo in Capis animadvertet. 
                    Sunt enim hi qui \ampersand\ Gallos \ampersand\ Gallinas, par amoris moderamine \edtext{persequuntur}{\Afootnote{prosecuntur}}, 
                    ac de gallis quidem non est quod dubitet quispiam, cum \ampersand\ ipsi aliquando galli fuerint, 
                    quod tu Euthyme paulo ante dicebas. 
                    pro Gallinis vero haec indicia sunt, quod olim mares eas communi instinctu dilexere, 
                    Et iam nunc cum excubatrix deficit, in amoris argumentum pullos fovent tactu subtus \edtext{iacentium}{\Afootnote{iacentum}} pullorum pruriente, quoque se oblectantes, 
                    semper deinceps, amant ducunt, pascunt, 
                    quod se expertum magnus naturae indagator, asseruit. 
                    Quae cum vera esse constet, nemo sane Capos in arbitratu suspectos habebit, utcumque cesserit sententia, 
                    Quam ego, ut jusserunt breviter \ampersand\ aperte vobis Spectatores
                    Audientibus partium patronis effabor.  
                     Gallinis enim \ampersand\ Gallis plura a Capis ipsis, iuste non nihil indignantibus, 
                    post sedatam hanc contentionem dicentur, maxime si ad aliorum nostrae provinciae aures haec res aperte pervenerit: 
                    ut certe pervenient. 
                    Quod enim iam sum locuturus, ex horum animis defluxit praecipue, quos hic mihi assidere videtis, 
                    qui nostri Ornithoboscii incolae, supremae \edtext{auctoritatis}{\Afootnote{autoritatis}} locum inter omnes obtinent, 
                    idque tenent memoria, quod a longis annis inter eos actum est, \ampersand\ per exercitam illam prudentiam facile quid futurum sit, 
                    odorati potuerunt, semper, ex longa \ampersand\ honestissima consuetudine, \ampersand\ libero provinciae suae decreto, 
                    in difficillimarum rerum \edtext{aestimatione}{\Afootnote{estimatione}} omnium consensu, primores. 
                    Hi igitur acclamantibus qui domi nostrae sunt reliquis, ingenue fatentur, fere omnia ab Gallis, 
                    Gallinisque gnaviter \ampersand\ attente agi, remque rusticam \ampersand\ villaticas divitias, per eos levissimis impensis augeri, 
                    Esse tamen, quod in utroque (ut imbecillitas fert mortalium) \edtext{vitio}{\Afootnote{vicio}} dare quis possit. 
                    Nimis enim morosas Gallinas, Gallos vero nimis petulantes, saepe animadvertisse asserunt, nec morigerantes semper Gallinas, nec Gallos semper tuentes. 
                    Quae quidem, cum \ampersand\ inter homines habeant libertatem impunem, quis gallis \ampersand\ gallinis imputabit? 
                    At istud consilium quo per invidiam correptae Gallinae, eo dementiae pervenerunt, 
                    ut Gallorum famam inter mortales illustrem admodum \ampersand\ prespectam, 
                    infringere atque infamem reddere, rebus non tam veris quam confictis \ampersand\ perperam excogitatis conatae fuerint, adeo displicet, 
                    ut quod maxime. 
                    Nullo enim instituto major ruina, rei suae publicae parari poterat, si eo voto ad metam perventum esset, quo res coepta est. 
                    Idcirco partium consiliis diligenter perpensis, cognitoque universo \edtext{negotii}{\Afootnote{negocii}} excursu, non possunt non asserere, Gallis strenuis illatam esse injuriam. 
                    Quod tu Philonice, ubi Gallinis dixeris ne \edtext{aegre ferant}{\Afootnote{aegreferant}}, oratas habeto, 
                    Apertae enim \edtext{justitiae}{\Afootnote{justiciae}} nemo commode pro   amicis etiam amicissimis, refragari potest, cumque justae poenae non nihil loci in Gallinis esse videretur, 
                    eam abolitam \ampersand\ expunctam in amoris sui argumentum volvere. 
                    Novistis legem, quae Iudices in levioribus causis vult esse ad lenitatem proniores, 
                    In gravioribus autem poenis, severitatem legum cum aliquo temperamento benignitatis subsequi. 
                    Capi causae nostra non judices sed Arbitri, poenam commissi, hercle non parui: 
                    non temperare, sed non reminisci volverunt Eius loco munus \edtext{exspectantes}{\Afootnote{expectantes}} pulcherrimum, \ampersand\ partibus commodissimum, 
                    Concordiam videlicet \ampersand\ matrimonii perenni pace stabiliti mite comertium, 
                    ad quod praestandum, tu Philonice, tuque Euthyme, cum domum venietis, Gallos, Gallinasque, iterum atque iterum hortemini. 
                    QUID est enim, Dii boni, in ea familia, qua vir \ampersand\ uxor rerum habent administrationem, divinius? 
                    Quid utilius pace \ampersand\ amore? 
                    Magnum legislatorem dixisse ferunt, Vitam reliquam minime videndam esse, quando uxor cum viro in discordia est. 
                    Ubi enim illa est morosa \ampersand\ jurgiosa invidia, Ubi irae \ampersand\ molestiae, per dies perque noctes scatent, 
                    Ibi de vita \edtext{quotidie}{\Afootnote{cotidie}} plus amittitur, adeo, ut mortuum esse satius esset, quam ad mortem tam lento \ampersand\ molesto passu pervenire. 
                    Monendae igitur Philonice sunt Gallinae, ut se suosque maritos ament, remque domesticam current, \ampersand\ invidiae errore conceptae stimulos, 
                    ex animis suis procul eliminent, sintque probae (Conjuge, namque proba nihil \edtext{exstat}{\Afootnote{extat}} dulcius usquam) \ampersand\ apertae, 
                    ut virorum suorum major per haec salus, majorque quies astruatur. 
                    Non frustra diligens quidam rei domesticae praeceptor dixit, 
                    Existimo probam conjugem, sociam domus esse, magnumque momentum ad viri \edtext{felicitatem}{\Afootnote{foelicitatem}}. 
                    Natura siquidem, quam primus ille mundi genitor perpetua \edtext{fecunditate}{\Afootnote{foecunditate}} donavit, operam mulieris ad domesticam diligentiam comparavit, 
                    Viri vero ad exercitationem extraneam, 
                    Itaque viro calores \ampersand\ frigora perpetienda, 
                    tum   etiam laborem pacis \ampersand\ belli distribuit, mulieri autem domestica \edtext{negotia}{\Afootnote{negocia}} curanda tradidit, 
                    quae ut curaret diligentius, viro timidiorem esse voluit. 
                    Nam metus ad diligentiam custodiendi plurimum confert nobilis agricolae testimonio, 
                    Perperam ergo invident Gallinae Gallis honestam illam, socialis pugnae exercitationem, \ampersand\ haec quasi militaria stipendia, 
                    quae in viris \edtext{rempublicam}{\Afootnote{rem publicam}} apud majores administrantibus raro desiderata sunt. 
                    Ex his enim exerciti propugnabunt audentius, \ampersand\ timebunt minus, 
                    quos tu Euthyme, tu Gallinas conjuges continenter ament curentque obsecrato, decet hoc eos, 
                    quia \edtext{reipublicae}{\Afootnote{rei publicae}} nostrae suo imperio aemuli cognoscuntur. 
                    Praeterea, ut omnium injuriarum oblivio sit. 
                    Fidus enim \ampersand\ sincerus amor ibi esse non potest, ubi injuriarum memoriam, latentibus scintillis invidia nutrit. 
                    Quod si pristinae benevolentiae favore, resipiscentes Gallinae, Gallique amantes consenserint, salvi erunt, vitamque agent beatam \ampersand\ utilissimam his, 
                    quorum \edtext{praediis}{\Afootnote{prediis}} suam istam civitatem locaverint. 
                    Nobilis Poetae sententia est. Nec divitias nec quicquam aliud tantum voluptatis habere, quantum virum \ampersand\ uxorem bonos. 
                    Eam si domi vestrae Gallis \ampersand\ Gallinis explicabitis sese amabunt, ut antea, 
                    Capisque agent, ut certe \edtext{debent}{\Afootnote{daebent}}, infinitas gratias, 
                    Ite igitur actutum domum, \ampersand\ concordiae consulite, 
                    in cuius solius incolumitate, Gallinarum \ampersand\ Gallorum sita est salus, dixi.
                \pend 
        \endnumbering
        \end{Leftside}
        \end{latin}

        \begin{Rightside}
        \beginnumbering
            \pstart\section*{Arbitrage des chapons, l'interprète Nomothetes parle.}\pend\pstart\subsection*{Nomothetes}\pend\pstart\textbf{Nomothetes}\hspace{1cm} 
                        Comme vous me l’avez ordonné, toi Philonique et toi Euthyme, j’ai exposé brièvement et sans artifice toute l’affaire aux chapons dont j’ignore s’ils ont jamais été aussi ébahi dans une autre affaire qu’ils furent aujourd’hui après qu’ils eurent entendu le désaccord des coqs et des poules.
                        Ils savent évidemment que rien n’est plus funeste qu’une guerre intestine  et que les fondements de la république dont les membres sont en désaccord avec la tête ne pourront pas tenir longtemps. Par ailleurs, l’historien amoureux de la vérité, dit que les petites choses croissent par l’harmonie et que les grandes choses s’évanouissent par la discorde. Ceci est si bien connu de vous que c'est ce qui l'est le mieux. 
                        À cause de cela (C’est pourquoi elle ne fut pas brièvement bienvenue pour ceux-ci) , il a semblé que l’instruction devait aller avec assez de précipitation vers la modération du pouvoir autant qu’il est dangereux d’avoir davantage de choses sur soi, si elle s’était éloignée très loin, elles remercient la bonne fortune et le destin favorable parce que la liberté parvint de bon plaisir vers eux par votre accord et celui des parties. Si en effet l’affaire avait marché autrement, il semblait les craindre, pour que la faible sécurité du genre n’aie pas commis quelque chose de plus affreux que de cette sorte, (séverine) il aura osé agir, parce que personne, par hercule, n’a fait attention pour les chapons.

                        En effet ce sont ceux qui poursuivent en justice et les coqs et les poules à l’aide d’une conduite égale de l’amour, et qui n’est certes pas venant des coqs parce qu’il balance entre deux choses, quelqu’un, puisque les coqs ont été à ce même individu, ce que toi Euthymus tu me disais un peu auparavant.
                        En vérité ces indices sont en faveur des poules, parce qu’elles ont jadis choisi les mâles par un instinct commun [eas ?], elles ont réchauffé les petits dans une preuve d’amour après avoir éprouvé une envie de toucher les petits couchés en dessous, aussi s’amusant, toujours en continuant, elles aiment, tirent, nourrissent, parce que [se] le grand investigateur a amené quelque chose de la nature qui a fait ses preuves.
                        Et puisque que c’est un fait établi que ces choses sont vraies, personne n’aura suspecté les Chapons de manière raisonnable dans la sentence de l’arbitrage, combien moi, lorsqu’ils ont ordonné, je vous dirai, spectateurs, brièvement et ouvertement après que vous avez entendu la partie des avocats.
                        En effet plusieurs venant des Chapons plaident pour les poules et les coqs, qui ne s’indignent justement pas pour rien, après cette lutte calme, principalement pour le cas où l’affaire serait parvenue ouvertement à ces oreilles d’autres personnes de notre province : puisqu’elles sont parvenues avec certitude.
                        Et en effet je suis déjà destiné à dire cela, qui a principalement coulé hors de l’âme de ceux-ci, que vous avez vu ici s’asseoir vers moi, qui (main)tiennent entre tous la situation de l’autorité de l’habitant suprême de notre poulailler, et ils tiennent cela dans leur mémoire, qui depuis de longues années s’est passé entre eux, et (joram) par cette prévoyance inquiétée qui sera facile, les parfumés purent, toujours, avec une longue et très honnête habitude, et par un décret libre de leur province et dans une estimation de toutes les affaires très difficiles par accord, devenir les nobles (les premiers). 
                        (Joram) Ceux-ci donc blâmant celles qui ont abandonnée notre maison, reconnaissent naïvement, presque tous des coqs, et d’avoir fait avec empressement et application à partir des poules, et un évènement champêtre et des fortunes des ferme, (et) d’avoir augmenté de lourdes dépenses à travers celles-ci, d’être cependant, parce que
                        
                    \pend
        \endnumbering
        \end{Rightside}
        \end{pages}
        \Pages
        \end{document}
    
